\documentclass[12pt,english]{article}
\usepackage[T1]{fontenc}
\usepackage[utf8]{inputenc}
\usepackage{graphicx}
\usepackage{float}
\usepackage[english]{babel}
\usepackage{a4wide}
\usepackage{xcolor}
\usepackage{listings}
\usepackage{tikz}
\usepackage{enumitem}
\usetikzlibrary{positioning, calc}
\usepackage{url}
\usepackage{csquotes}
\usepackage{verbatim}
\usepackage{newfloat}
\setlength{\marginparwidth}{2cm}
\usepackage{todonotes}
\usepackage{indentfirst}
\usepackage{svg}
\usepackage{caption}
\usepackage{fvextra}
\usepackage{xspace}
\usepackage{lipsum}

\usepackage{newcomputermodern}

% needed for breaking in \texttt
% https://tex.stackexchange.com/questions/299/how-to-get-long-texttt-sections-to-break
\usepackage[htt]{hyphenat}


\definecolor{pblue}{rgb}{0.13,0.13,0.7}
\definecolor{pgreen}{rgb}{0.0,0.5,0.0}
\definecolor{pred}{rgb}{0.8,0.0,0.0}
\definecolor{pyellow}{rgb}{0.55,0.33,0.03}
\definecolor{pgrey}{rgb}{0.5,0.5,0.5}
\definecolor{codebg}{rgb}{0.98,0.98,0.98}
\definecolor{pblack}{rgb}{0.1,0.1,0.1}

\usepackage[newfloat]{minted}
\setminted{bgcolor=gray!10, breaklines, fontsize=\small}
\newcommand{\mintedlabel}[1]{\label{line:#1}}

\usepackage{xcolor}

%% needed for minted label
\newenvironment{code}{\captionsetup{type=listing}}{}

% THEME DOES NOT WORK

% Define the TempleOS-inspired color palette
\definecolor{templeos-bg}{HTML}{0000AA}
\definecolor{templeos-fg}{HTML}{FFFFFF}
\definecolor{templeos-cyan}{HTML}{00FFFF}
\definecolor{templeos-green}{HTML}{00FF00}
\definecolor{templeos-yellow}{HTML}{FFFF00}
\definecolor{templeos-red}{HTML}{FF0000}
\definecolor{templeos-gray}{HTML}{AAAAAA}

\newminted{c}{
    bgcolor=templeos-bg,
    fontsize=\small,
    linenos=true,
    frame=lines,
    framesep=2mm,
    fontfamily=courier,
    commentstyle=\color{templeos-green},
    keywordstyle=\color{templeos-cyan},
    stringstyle=\color{templeos-red},
    numberstyle=\color{templeos-gray},
    identifierstyle=\color{templeos-fg},
    functionstyle=\color{templeos-yellow},
    outputdir=./,
    breaklines
}

\lstdefinestyle{inlinecommandcore}{
    basicstyle=\ttfamily\small\color{black}, % <--- ADD \color{black} HERE
    keepspaces=true,
    showspaces=false,
    showstringspaces=false,
    language=bash,
    morekeywords={sudo, arpspoof, apt, install, ipconfig, arp, ifconfig, netstat, sysctl, ping},
    keywordstyle=\color{blue},
    commentstyle=\color{green!60!black},
    stringstyle=\color{purple},
    literate={>}{{\textgreater}}1 {<}{{\textless}}1 {\_}{{\textunderscore}}1,
    columns=flexible,
    basewidth={0.5em,0.4em},
    aboveskip=0pt,
    belowskip=0pt,
    literate={``}{{"}}1 {''}{{"}}1,
}

% Define the command combining boxing and listing
\usepackage{xparse}
\NewDocumentCommand{\inlinecode}{v}{%
    \begingroup
    \setlength{\fboxsep}{1.5pt}%
    \setlength{\fboxrule}{0.4pt}%
    \textcolor{black!40}{%
        \fbox{%
            \colorbox{black!5}{%
              \lstinline[style=inlinecommandcore]{#1}%
            }%
        }%
    }%
    \endgroup
}



\usepackage[backend=biber,style=ieee,language=english]{biblatex}
\addbibresource{oshell.bib}

%% remove for default font
%% \usepackage{fontspec}
%% \setmainfont{EB Garamond}
%% \usepackage{libertinus-otf}


% three custom commands to create an unnumbered (sub(sub))section
% that still gets included in table of contents.

\newcommand{\usection}[1]{\section*{#1}
\addcontentsline{toc}{section}{\protect\numberline{}#1}}

\newcommand{\usubsection}[1]{\subsection*{#1}
\addcontentsline{toc}{subsection}{\protect\numberline{}#1}}

\newcommand{\usubsubsection}[1]{\subsubsection*{#1}
\addcontentsline{toc}{subsubsection}{\protect\numberline{}#1}}

\newcommand{\funcname}[1]{\texttt{#1()}\xspace} % for functions

\newcommand{\projectname}{\textsc{oshell}}


\graphicspath{{figures/}}
\svgpath{{figures/}}

\newcommand{\mycomment}[1]{}


\usepackage[hidelinks]{hyperref}
\usepackage{cleveref}
\crefname{listing}{listing}{listings}
\begin{document}

\pagenumbering{gobble}

%%%%%%%%%%% TITLE PAGE %%%%%%%%%%%

% Recommended: German Academic Style (English Document)
% Really good

\mycomment{
  \begin{titlepage}
\begin{center}
	\vspace*{30mm}

	% University/Institution at top (German style)
	{\large
	Universität Ulm\\
	Faculty of Engineering, Computer Sciences and Psychology\\
	Institute of Distributed Systems
	}\\

	\vspace{25mm}

	% Main title
	{\fontsize{12mm}{14mm}\selectfont\bfseries
    oshell}\\
	\vspace{8mm}

	% Subtitle
	{\Large Implementation of a Unix Inspired Shell in C11}\\

	\vspace{15mm}

	% Document type (common in German academia)
	{\large Project Report / Bachelor Thesis / Seminar Paper}\\

	\vspace{30mm}

	% Author information in German academic format
	\begin{tabular}{ll}
	\textbf{Author:} & Your Full Name \\[3mm]
	\textbf{Matriculation Number:} & 123456 \\[3mm]
	\textbf{Email:} & your.email@uni-ulm.de \\[8mm]
	\textbf{Supervisor:} & Prof. Dr. Supervisor Name \\[3mm]
	\textbf{Second Examiner:} & Dr. Second Examiner \\[8mm]
	\textbf{Submission Date:} & \today \\
	\end{tabular}

	\vfill

	% Semester info at bottom (German style)
	{\normalsize Winter Semester 2024/25}

\end{center}
\end{titlepage}

}

% Hauptseite für erstmal
\begin{titlepage}
\begin{center}
    \vfill
	
	% Title with sans-serif font
	{\sffamily\fontsize{16mm}{18mm}\selectfont\bfseries
    oshell}\\
	\vspace{12mm}

	% Subtitle in smaller sans-serif
	{\sffamily\large Implementation of a Unix Inspired Shell in C11}\\

	\vspace{60mm}

	% Clean author and date layout
	\begin{tabular}{ll}
	\textbf{Author:} & Ole Wortmann \\[5mm]
	\textbf{Date:} & \today \\[5mm]
	%%\textbf{Version:} & 1.0 \\
	\end{tabular}

	\vfill

\end{center}
\end{titlepage}

\newpage
\begin{abstract}
  \textit{
\textbf{abstract}
  }
\end{abstract}
\newpage

\pagenumbering{roman}
\tableofcontents

\newpage
\pagenumbering{arabic}
\section{Introduction}
The following is a project report for a small Unix-based
shell made by \{the author\} was not created in any official
context and should be treated as such.

This project came to life because I was looking for a
new project that was not game developement, had something to
do with systems but was not as complicated as making an
Operating system.
% look over this, tho I like it it does not really fit
Tho this may very well be a step in the
direction of starting developing one, there are no plans as
of right now.
%%%
Since it had peaked
my interest in the past, but I was just not ready implementing
the C standard library.

This is neither something new
nor outstanding but just a fun project that took me way
longer then I would like to admit, because I only thought
of some edgecases or obvious failure points, after I was
already ``done''.

It also does not implement any new ideas, but builds on a
specification sheet, which you will learn more about in the
planning section (see~\Cref{sec:planning}). It also mostly
follows the bash shell way of commands, tho keep in mind
that this is really simple, and advanced bash commands might
not work. At the time of writing this command chaining (\&\&)
aswell as background execution (\&) have not (yet) been implemented.


\section{Planning} \label{sec:planning}
Since building a \textit{(small)} shell is not unusual for a University
project, I figured there would be some article or specification or
anything that could give me a sense of direction and ease the
planning. I found a specification from a professor at Florida
State University (FSU), but since I cannot find the article in
which it was posted I can only provide a direct download link
(\url{https://www.cs.fsu.edu/~cop4610t/assignments/project1/writeup/specification.pdf}
. A copy is provided in the github repository.

The specification provided a clear roadmap to follow outlining
features and functionalities to implement.
% Maybe what features and functionalities are to implement
I did not follow it linear, but
just worked on what I felt like. 

%% problem → options considered → final choice → justification.

\section{Essential Terminology and Concepts}
This will explain some basic like \textsc{STDIN/STDOUT}, file
descriptors and other terms that will come up in this
report.
\begin{itemize}
\item STDIN: Standard input (integer value \textsc{0}), comes
  from a ``standard input device'' which is usually
  (but not limited to) the keyboard, but could also be a
  \texttt{file.txt}.
\item STDOUT: Standard output (integer value \textsc{1}),
  either goes to the terminal or an X terminal, this 
  entirely depens on where the process originated.
\item File Descriptors: ``A file descriptor is an unsigned
  integer used by a process to identify an open file'' \cite{AIX2025}. An open file could be a normal (text) file or
  \texttt{STDIN}/\texttt{STDOUT}.
\end{itemize}~\cite{StdinStdoutStderr}

\section{Design}
\mycomment{
  \textbf{Checklist for design section}
\begin{itemize}
\item Maybe some high level explanation flow chart of everything
\item Go into detail with inbuilt commands and piping redirection \ldots
\item CHOOSE MOST COMPLEX/INTERESTING PART to go indepth, flow chart explanations \ldots, flow chart explanations \ldots
\item \textbf{Focus on the overall structure and flow of your program, rather than specific code details}
\end{itemize}

}

% what gets extracted during parsing (maybe)
\textbf{\projectname} follows this really simple \texttt{read-parse-execute} loop, \Cref{fig:mainLoopFlowchart}
displays this nicely. Input
is parsed, and then passed to the \funcname{execute\_command} function. This is really the most obvious way to
implement the high level behavior of a shell, since it is always running, and always needs to do these three things.

\begin{figure}[!htb]
  \centering
  \includesvg[scale=1]{mainLoopFlowchart.drawio}
  \caption{Figure visualising the main flow}~\label{fig:mainLoopFlowchart}
\end{figure}

The \textbf{input handler} passes the string without modification to the
\textbf{parser}, which parses the string, tokenizes it, and removes exces quotes.
The command execution function then checks for different character ``operators'' in the parsed string,
to then trigger different functions based on the input. 

\Cref{fig:executeCommandFlowchart} shows a
simplified flowchart of what the execution looks like.
\begin{figure}[!htb]
  \centering
  \includesvg[scale=1]{executeCommand.drawio}
  \caption{Figure visualising the flow of command execution (simplified)}~\label{fig:executeCommandFlowchart}
\end{figure}

The specification sheet, specified the commands which were to be implemented as inbuilt
(although not all have been).

\todo{memory leaks still exist so this will not be written yet.}
\lipsum[1]
% Your approach to memory management (dynamic allocation, cleanup strategy)
\newline
\todo{this is also a todo}
% A brief mention of your overall error handling approach - do you try to recover gracefully, provide user-friendly
% messages, etc.?
\lipsum[1]

\section{The Implementation}
\subsection{Getting the input}
C provides several functions for capturing
the user input (\texttt{STDIN}) from the commandline, including \funcname{fgets}
(\url{https://linux.die.net/man/3/fgets}), and
\funcname{getline} (\url{https://linux.die.net/man/3/getline}).
While both are valid options to capture user input, \funcname{getline} is (in this case) superior.

While initially \funcname{fgets} seemes like a valid
approach, it stops when reading a newline character
(\texttt{'\textbackslash n'})
and also does not handle memory management by itself~\cite{OS2023}.
Since \textbf{\projectname} does not support command chaining
by \&\& this is the only way to chain commands together e.g.
\mintinline{bash}{command \n command1 \n command2}

The \funcname{getline} function does all this
, it reads line-by-line, includes the newline
character, null-terminates the buffer and
handles the memory management for the buffer.

% this is shit, but I need some 3-4 line text to get the codeblock to the next page (elegantly)
\mycomment{
  Which is exactly what you want, because this means that the input handling is less prone to (memory) errors then it
would
be with any other input reading function.

}
\todo{i feel like this is missing something, there is no text here}
\begin{minted}[linenos]{c}
char *oshell_read_line(void)
{
    char *line = NULL;
    size_t buffsize = 0;

    if(getline(&line, &buffsize, stdin) == -1)
    {
        if(feof(stdin))
        {
            free(line);
        } else {
            perror("oshell: readline");
            free(line);
            exit(EXIT_FAILURE);
        }
    }
    return line;
}
\end{minted}

\subsection{Parsing the user input}
Parsing the input for a small/basic shell might not seem like
a huge endevour, just look at the first part of the input, as the command
and the second as the argument, pass that to \funcname{execv}
(\url{https://linux.die.net/man/3/execv}) and you are done. This works for something like \mintinline{bash}{echo "hello world"},
but this will fall apart quite fast for anything even slighty more complex e.g.
\begin{minted}{bash}
echo "foo bar baz" | wc -w
\end{minted}
\textbf{Expected Output:}
\begin{minted}[bgcolor=gray!10]{text}
3
\end{minted}
Because now the pipe operator \texttt{(|)} needs to be dealt with (see~\Cref{sec:piping}).
To successfully parse the user input we need to:
% explain what tokens are
% why do we need to tokenize
% (how do we tokenize)?

% why do we replace the last char if newline char with null-terminator
% why do we need to remove the quotes? Why do we remove them very last?
% echo "test 1 2 3" -> ["test 1 2 3"] is not the same as [test] [1] [2] [3]

\begin{enumerate}
\item Tokenize
  \begin{itemize}
  \item What are tokens?
    Tokens are the smallest ``unit'' that the shell can work with. When
    looking at a command like \mintinline{bash}{ls -l /home/user} the tokens
    would be [\texttt{ls}] [\texttt{-l}] [\texttt{/home/user}].

  \item Why is tokenization necessary? Tokenization is needed because
    the \funcname{execv} function that \textbf{\projectname} uses to
    execute the
    commands, takes the \texttt{path} and the \texttt{argv} seperate.
    \todo{this is also needed for parsing, and actually called in parse function}

  \item How does the tokenization work? When tokenizing
    \textbf{\projectname} loops through the input string and
    checks for every character if it is a space or a null-terminator \texttt{('\textbackslash0')}.
    They act as so called \textbf{delimiters},
    ``A delimiter is one or more characters that separate text strings.''~\cite{WhatDelimiter}.
    If the character 
    to be replaced is not the first character, it is replaced with the \texttt{('\textbackslash0')},  to terminate the string. The adress of the beginning of the new string is then saved to a \texttt{tokens} array, and if needed 
   the memory will be reallocated to fit the string.
    \end{itemize}
    
  \item If the last character is a newline, replace it with null-terminator
\begin{minted}[linenos]{c}
char *mod_str = args[i];
if (strlen(mod_str) > 0 && mod_str[strlen(mod_str) - 1] == '\n') {
    mod_str[strlen(mod_str) - 1] = '\0';
}
\end{minted}
    \captionof{listing}{codeblock from parse function showing way of shortening string but not always?}~\label{mnt:WhatShows}

    If the last character of the last token
\texttt{mod\_str[strlen(mod\_str) - 1]} is a newline~\texttt{(' \textbackslash n')},
it gets replaced with a null-terminator \texttt{('\textbackslash 0')}. 
This can effectively ``shorten'' the string. 
Which would look something like this: 
\[
\texttt{foobar\textbackslash n\textbackslash 0} 
\;\rightarrow\; 
\texttt{foobar\textbackslash 0\textbackslash 0} 
\;\rightarrow\; 
\texttt{foobar\textbackslash 0}
\]

This sanitization is necessary because some programs may behave
incorrectly or crash
if they receive strings with embedded newline characters.


\item Removing the outer quotes after
  tokenization is necessary so that the command and thus
  the output (mostly)
  behaves like assumed.
  This is a ``very basic'' implementation of parsing quotes and
  building new strings. As of writing this \textbf{\projectname}
  does not behave 1:1 like \textsc{bash} or \textsc{zsh}, this
  can be seen with the parsing and handling of newline characters
  inside quotes. Most simple example would be \texttt{echo} and
  \texttt{sort} e.g.
\begin{minted}[linenos]{bash}
echo "3\n1\n2" | sort -n
\end{minted}
  \textbf{Expected Output}
\begin{minted}[linenos]{bash}
tuxpad% echo "3\n1\n2" | sort -n
1
2
3
\end{minted}
  But \textbf{\projectname} will just parrot the string back to you because as
  of writing this, it does not handle the newline character inside
  quotes as the newline character but just as a string literal.
\begin{minted}[linenos]{bash}
echo "3\n1\n2" | sort -n
3\n1\n2
\end{minted}
  providing echo with the \texttt{-e} flag, will yield the correct
  output.
\begin{minted}[linenos]{bash}
echo -e "3\n1\n2" | sort -n
1
2
3
\end{minted}
  Using
  \mintinline{bash}{echo "test123" >> append.txt} as an example,
  if during parsing the quotes are not removed the quotes will
  be appended to the file so
  \mintinline{bash}{cat append.txt} would return
\begin{minted}{text}
"test123"
\end{minted}
  instead of
\begin{minted}{text}
test123
\end{minted}
\end{enumerate}
  \subsubsection{Removing quotes and building a new string}
  The \funcname{remove\_quotes}
  function (see~\Cref{mnt:removeQuotes}),
  \mycomment{
 \textbf{---------------}
  keeps track of the
  processed arguments to be able to piece together the
  ``unmodified'' strings later.
\textbf{---------------}
  }
  loops through every word, and letter, to find either a first
  single or double quote. Uppon finding one, it logs the first word
  and letter position and the last word and
  letter position (See~\Cref{mnt:removeQuotes}
  lines 28~-~46).
  If a closing quote is found, the \funcname{build\_quote\_string}
  function is called with the following parameters
  provided,
  \texttt{arg, start\_arg, start\_pos, end\_arg, end\_pos}.
  This then just turns the multiple arguments into one string without
  the quotes, null-terminating the string (See~\Cref{mnt:bqString}).

\subsection{Executing Commands}
% THIS READS LIKE A FIFTHGREADER GO INTO MORE DETAIL
% WRITE BETTER
The \funcname{execute\_command} function uses the return value
\texttt{char **args} from the \funcname{parse} function which
holds the
command name, and the arguments e.g.
\mintinline{bash}{echo "hello world"}.

It first checks for the inbuilt commands such as,
\texttt{exit}, \texttt{help}, \texttt{cd} or \texttt{kill}, which \funcname{exit}
the programm print some help text, change the directory or
kill a process based on the \textsc{PID} or
\texttt{process name}. \textbf{\projectname} can also do
piping, output and input
redirection, this is all handled internally before spawning
any external processes.

If the command or the arguments do not call any of the
internal functions, \textbf{\projectname} builds the command with the
proper structure, and creates a fork to try and execute the
command.

The following subsections will go into more detail about
how each function/system works.

\subsection{Builtin Commands and functionalities}
\subsubsection{CD - Change Directory}
The inbuilt commands all follow the same pattern,
\funcname{strcmp} checks if the command
(\texttt{args[0]}) is equal to any of the inbuilt commands. Should this be the case
it calls the correct function. \newline
E.g. ``cd'' there are two cases:
\begin{enumerate}
\item User wants to navigate to directory X. \texttt{args[1]} is
  not \textsc{null} meaning there is something behind \texttt{cd}
  ideally
  a path to a directory
\item User wants to take the shortcut to \texttt{\$HOME}. \texttt{args[1]} is \textsc{null}.
\end{enumerate}

\begin{minted}[linenos]{c}
if (strcmp("cd", command) == 0)
    {
        if(args[1] != NULL) {
            int status = change_directory(args[1]);
            if (status != 0) {
                perror("oshell: change_directory() error");
                return 1;
            }
        } else {
            chdir(getenv("HOME"));
        }
        return 0;
    }
\end{minted}
\captionof{listing}{sourcecode showing the 'cd' implementation of
\projectname}
\label{mnt:cd}

 Depending on if \texttt{args[1]} is \textsc{null} or not the  
\funcname{change\_directory}
(see~\Cref{mnt:changeDir}) function is called
and the wanted directory name is passed to it. If the user
enters a directory name that does not exists in the directory the user is currently in, an error is thrown and the user is
sent back to the commandline. Depending on
if the first character after the \texttt{cd} command is a
tilde (\texttt{\string~}), a new path is constructed
with \texttt{\$HOME/user/\emph{some/new/path}} because the \funcname{chdir}
function, \textbf{\projectname} uses to actually change
the directories,
does not use relative paths with tilde, but only
absolute paths.

\subsubsection{Kill - Killing a process by name or ID}
\todo{why might you want to kill a process, and not just press the
close window icon?}
Killing a process by \textsc{id} or \textsc{name} is done
through the same \funcname{kill\_process} function.
This function takes a \texttt{char *process\_name\_or\_id},
and utilizes a helper function called
\funcname{string\_to\_int}
that can parse the \texttt{process\_name\_or\_id}
into an integer if the string represents a valid number.
Based on the return value of the \funcname{strint\_to\_int}
function it will the two cases:
\begin{enumerate}
\item User submitted a valid integer
 \item User submitted a string
 \end{enumerate}

 In the event that the user submitted an \texttt{int} 
 it will be cast to \texttt{pid\_t}
 and passed to the \funcname{kill}
 (\url{https://linux.die.net/man/1/kill}) function.

 In the event that the user submitted a string, the function will
 try and get the \texttt{PID} it does this the following way.

 On unix systems there is a pseudo-filesystem called
 \texttt{/proc/} which acts like an \texttt{API} to kernel data structures.
 It holds a subdirectory to each process that is currently running.
 Each of those holds process specific files, like the
 \texttt{comm}
file~\cite{Proc5LinuxManual}~\cite{ProcPseudoFile}.
The \texttt{comm} file stores the process name e.g.
\mintinline{bash}{cat/<PID>/comm} could return ``zsh'',
 and has the \texttt{PID} of the running process, as the
 filename.
 When looping through \texttt{/proc/<PID>} the programm
 checks whether the directory name is all integers, if so
 it constructs a path to said folder.
\begin{minted}[linenos, breaklines]{c}
snprintf(full_proc_path, full_proc_pathLength ,"%s%s", proc_dir_name, dirent->d_name);
\end{minted}
 Else it just takes the \texttt{PID} provided and tries
 to kill the process.
 
\begin{minted}[linenos, breaklines]{c}
if(strcmp(strip_non_alpha(buffer), strip_non_alpha(process_name_or_id)) == 0) {
    if(kill(pid, 9) == 0) {
        fprintf(stderr, "killed %s with PID: %i", process_name_or_id, pid);
        fflush(stderr);
        process_found = true;
    }
    goto cleanup;
}
\end{minted}
\captionof{listing}{sourcecode showing comparison between file contents
(buffer) and the user input}

\medskip
 Before validating if this is the correct process
 the \texttt{PID} (directory name) is saved, so that it can be
 later used to kill said process if it is the correct one.
 A new path to the \texttt{comm} file is then created to
 be able to read the file contents, if the contents and the
 user input match the process is killed.

 \subsubsection{Piping} \label{sec:piping}
The implementation of piping
(See~\Cref{mnt:pipeRedirection})
was one of the hardest functionalities to implement.
A pipe is a point-to-point connection that has one read-end and one
write-end. It allows for the standard output of one process
to become the standard input of another process, without needing to
make any temporary files or something which makes it extremly
efficient.

\begin{figure}[!htb]
  \centering
  \includesvg[scale=1]{pipeVis.drawio}
  \caption{Figure visualising the pipe}~\label{fig:pipeVis}
\end{figure}

%% This is not smooth and needs a redo
The string is divided into multiple strings whenever the 
pipe operator \texttt{(|)} is identified, and the position of it is saved.
The pipe positions are used to be able to construct 
\texttt{commands} out of the seperated strings e.g.
\mintinline{bash}{ls -la | grep txt | wc -l}
is represented as three different commands, always split at the
pipe symbol.

\medskip
Each command is executed as its own process. Using \funcname{fork}
and \funcname{execv} while they are ``connected'' to the pipe.
Depending on the commands position the the content is redirected
different.

\begin{itemize}
\item First command: \textsc{STDOUT} is redirected to the \textsc{STDIN} of the following command.
\item Middle command(s): \textsc{STDIN} will receive the
  \textsc{STDOUT} of the previous command, and will redirect
  the output to the \textsc{STDIN} of the following command.
\item Last command: Gets its \textsc{STDIN} from the previous
  command it points its \textsc{STDOUT} to the current terminal.
  \end{itemize}

\Cref{fig:multipipe} shows a visualisation of the data flow.
  
\begin{figure}[!htb]
  \centering
  \includesvg[width=0.9\linewidth]{multiPipe.drawio}
  \caption{Figure visualising the data flow of multiple pipes chained together}~\label{fig:multipipe}
\end{figure}

The parsed commands are represented as
\mintinline{c}{char ***commands}, which is simply a pointer to an array
of pointers that point to pointers of characters (a string). Which is a typical way to
represent arguments for a shell since this can store mulitple
arguments while still being able to easily access individual strings if needed. \Cref{fig:memoryLayout} gives a simplified
view of how \mintinline{bash}{echo "foo bar baz" | wc -w} is
mapped in memory.

\begin{figure}[!htb]
  \centering
  \includesvg[width=0.7\linewidth]{pipeRedirectionMemoryLayout.drawio}
  \caption{Memory layout example of \texttt{commands} (simplified)}~\label{fig:memoryLayout}
\end{figure}

\subsubsection{Clear - Clearing the terminal}
The \funcname{clear} function does not actually erase the content. Instead, this function prints enough newline characters to move the existing text beyond the visible
screen area. The previous content remains in memory and can be
accessed by scrolling up.

\medskip
It works by calling the \funcname{ioctl} function, with
the open file descriptor, the request type and in this case
to store the output in
the \texttt{winsize structure} (\texttt{w}).
The \texttt{0} (\textsc{STDIN}) tells the \funcname{ioctl}
function that the terminal it wants information
about
is the one currently being used.

\medskip
The \texttt{TIOCGWINSZ} is a request that returns the current
windowsize~\cite{UbuntuManpageIoctl_list}. It then loops over
the amount of rows the
terminal currently has, and prints a newline character
each time, after that the cursor is being moved to the top
left. 

\begin{minted}[linenos]{c}
void clear(void)
{    struct winsize w;
    ioctl(0, TIOCGWINSZ, &w);

    for (int i = 0; i < w.ws_row; i++) {
        fprintf(stderr, "\n");
    }
    fprintf(stderr, "\033[0;0H");
}
\end{minted}
\captionof{listing}{sourcecode showing implementation of the
\funcname{clear} function}
\label{mnt:clear}
\subsubsection{Input Redirection (<)}~\label{sec:inputRedirection}
Input redirection redirects the input
for the command. It allows the command to read the input
from a file instead of the standard input.
This is esentially a toned down version of piping
(see~\Cref{sec:piping}). Opposed to
piping it does not work with live processes, only with files
, this makes it usefull for simple operations.

\medskip
So instead of feeding the information
in the commandline like so: \mintinline{bash}{echo "Hello World" | tr 'a-z' 'A-Z'},
it can be read from a file e.g.
\mintinline{bash}{tr 'a-z' 'A-Z' < test.txt}. 
This will open \texttt{test.txt} and redirect its contents to the \textsc{STDIN} of \texttt{tr}.

 \Cref{fig:inputRedirection} shows a visualisation of how input redirection might look like.
\begin{figure}[!htb]
  \centering
  \includesvg[width=0.7\linewidth]{inputRedirection.drawio}
  \caption{Figure to help visualize input redirection}~\label{fig:inputRedirection}
\end{figure}

For the implementation see
\Cref{mnt:inputRedirection}.

\subsubsection{Output Redirection \texorpdfstring{($>$, $>>$)}{(>, >>)}}\label{sec:outputRedirection}
% Find a good example for output redirection
A user can redirect output from a command like \texttt{ls} and
either truncate (\texttt{$>$}), or append (\texttt{$>>$}) to the file. This works the same as
input redirection, only for \texttt{STDOUT}.
% How to include (better) example smoothly

\mintinline{bash}{ls > ls.txt} will create the file if needed
and truncate if exists.


\mintinline{bash}{echo "line1\nline2\nline3" >> append.txt}
will create the file and append if exists.
\Cref{fig:outputRedirection} shows a simple visualisation of how
the output redirection might look like.

\begin{figure}[!htb]
  \centering
  \includesvg[width=0.8\linewidth]{outputRedirection.drawio}
  \caption{Figure to help visualize output redirection}
  \label{fig:outputRedirection}
\end{figure}

For the implementation of this see \Cref{mnt:outputRedirection}

\subsection{Environment Variables}
An environment variable is basically a variable, with
a name and value. They can hold all sorts values, for example
the \texttt{HOME} path, things like the
\texttt{DESKTOP\_SESSION} or your current \texttt{SHELL}.
On Linux they can be listed with \texttt{printenv} or
\texttt{env}.
\textbf{\projectname} uses the \funcname{secure\_getenv} (\url{https://linux.die.net/man/3/secure_getenv})
function to retrieve the value behind the variable if valid.

\begin{minted}{c}
int dollar_pos = find_shell_operator("$", args);
if(dollar_pos >= 0 && args[dollar_pos][1]) {
    char envChar = args[dollar_pos][0];
    strncpy(envVar, &args[dollar_pos][1], sizeof(args[dollar_pos] -1));
    char envCharStr[2] = {envChar, '\0'};  
    if(strcmp("$",envCharStr) == 0) {
        fprintf(stderr, "%s", secure_getenv(envVar));
        free(envVar);
        return 0;
    }
}
\end{minted}
\captionof{listing}{sourcecode showing implementation of
\textbf{\projectname} handling environment variables.}
\label{mnt:envVar}
\todo{should the code be explained a bit (better)}
\medskip
The \funcname{find\_shell\_operator} function loops
through the string array, provided to it and checks against
the operator given (see~\Cref{mnt:findOperator}).

\medskip
Since \funcname{find\_shell\_operator} will only return
a positive non zero value on success it is safe to assume
that if the programm goes inside the \texttt{if} statement,
there is at least once char behind the dollar sign
(\texttt{\$}). This means
that we can copy everything but the dollar sign
and construct an environment variable to pass to the
\funcname{secure\_getenv} function.

\todo{better title needed, also not final this can do better}
\subsection{Input loop}
The shell is something that needs to continuously run,
to read user input, parse it manage different processes and execute
commands.

\textbf{\projectname} implements this using the
\funcname{oshell\_loop} as its main function. This basically runs
the whole shell.
\begin{enumerate}
    \item \textbf{Read}: Reads the user input line by line.
    \item \textbf{Parse}: Tokenizes the string and parses it
      e.g. for operators like \string<, $<<$, \string~ or
      \string$
    \item \textbf{Execute}: Based on the input of the string
      performs different actions to execute the command,
      inbuilt commands are handled differently then commands
      that have a pipe or input redirection.
\end{enumerate}

\Cref{mnt:oshellLoop} shows the implementation of the main loop.
\begin{minted}[linenos]{c}
void oshell_loop(void)
{
    char *user_input = oshell_read_line();
    char **args = parse(user_input);

    execute_command(args);
    free(user_input);
    free(args);
}
\end{minted}
\captionof{listing}{sourcecode showing the main loop}
\label{mnt:oshellLoop}

\section{Testing}
Testing and Validation: A critical part of any software project is demonstrating that it works correctly. A dedicated section on how you tested your shell is important. This could include:

\begin{itemize}
\item Unit testing: Did you test individual functions?

\item Integration testing: Did you test how different parts work together?

\item User acceptance testing: Did you run it with a set of common commands to verify it functions as expected?
\end{itemize}
\section{??Evaluation??}

\newpage
\usection{Bibliographie}
\printbibliography
\newpage
\pagenumbering{gobble}
\appendix
\usection{Appendix}
\usubsection{Parsing - Remove quotes function}
\begin{minted}[linenos, breaklines]{c}
char **remove_quotes(char **arg) {
    size_t arg_count = 0;
    size_t new_arg_count = 0;
    int quote_string_pos = 0;
    while(arg && arg[arg_count]) arg_count++;
    char **tmp_args = malloc((arg_count + 1) * sizeof(char*));
    if(tmp_args == NULL) perror("tmp_args memory allocation failed ");

    // Initialize the last element to NULL
    tmp_args[arg_count] = NULL;


    bool *processed = calloc(arg_count, sizeof(bool)); // Track which args are processed

    // this is not declared with the others e.g. start_arg,start_pos 
    // because it is needed later down out of 
    // that scope
    int end_arg = -1;
    int end_pos = -1;
    
    for(int i = 0; i < arg_count; i++) {
        if (processed[i]) {
            continue; // Skip already processed arguments
        }

        bool found_quote = false;

            if ((arg[i][j] == '"' || arg[i][j] == '\'')) {
                char quote = arg[i][j];
                int start_arg = i;
                int start_pos = j;

                // Search for closing quote
                int current_i = i;
                int current_j = j + 1;

                bool found_closing = false;
                while (current_i < arg_count && arg[current_i] != NULL) {
                    while (current_j < strlen(arg[current_i])) {
                        if (arg[current_i][current_j] == quote) {
                            end_arg = current_i;
                            end_pos = current_j;
                            found_closing = true;
                            break;
                        } 
                        current_j++; 
                    }
                    if (found_closing) break;
                    current_i++;
                    current_j = 0;
                }
                
                if (found_closing) {
                    // Build the quoted string and store it in new_args[i]
                    tmp_args[i] = build_quote_string(arg, start_arg, start_pos, end_arg, end_pos);
                    // fprintf(stderr, "quoted string [%i]: %s\n", i, tmp_args[i]);

                    // Mark all arguments from start_arg to end_arg as processed
                    for(int k = start_arg; k <= end_arg; k++) {
                        processed[k] = true;
                    }

                    found_quote = true;
                    break; 
                } else {
                    fprintf(stderr, "Warning: No closing quote found for quote starting at arg[%d][%d]\n", start_arg, start_pos);
                    break; 
                }
            }
        }

        // If no quote was processed, copy the original string
        if (!found_quote) {
            // fprintf(stderr, "no found quote [%i]: %s\n", i, arg[i]);
            tmp_args[i] = malloc(strlen(arg[i]) + 1);
            if(tmp_args[i] == NULL) fprintf(stderr,"oshell: memory allocation for %s failed", tmp_args[i]);
            strcpy(tmp_args[i], arg[i]);
        }

        int n = arg_count + 1;

        int j = 0;
        for (int i = 0; i < n; i++) {
            if (tmp_args[i] != NULL) {
                if (i != j) tmp_args[j] = tmp_args[i];
                j++;
            }
        }
        // Set leftover slots to NULL
        for ( ; j < n; j++) tmp_args[j] = NULL;
    }
    free(processed);
    return tmp_args;
}
\end{minted}
\captionof{listing}{Removes quotes from argument strings.}
\label{mnt:removeQuotes}

\newpage
\usubsection{Execute Command - Change Directory}
\begin{listing}[!htb]
\begin{minted}[linenos]{c}
int change_directory(char *directory)
{
    // add /home/user/ if ~ is the first char
    char* new_path = NULL;
    bool use_home_path = false;
    if(strcmp(&directory[0], "~") == 0) {
        char *user_path = getenv("HOME");
        int new_path_length = strlen(user_path) + strlen(directory) + 1;
        // first char is ~ replace dir with /home/user/dir
        remove_char(directory, '~');
        // fprintf(stderr, "%s%s",user_path, directory);
        new_path = malloc(new_path_length);
        snprintf(new_path, new_path_length, "%s%s", user_path, directory);
        fprintf(stderr, "%s\n", new_path);
        use_home_path = true;
    }
    int res = 0;
    if(use_home_path) {
        res = chdir(new_path);
    } else {
        res = chdir(directory);
    }
    if (res == -1) {
        free(new_path);
        return -1;
    }
    free(new_path);
    return 0;
}
\end{minted}
\label{mnt:changeDir}
\caption{sourcecode showing the implementation of the
change\_directory function}
\end{listing}

\newpage
\usubsection{Execute Command - Kill}
\todo{restore some comments, but remove commented functions and clutter.}
\begin{minted}[linenos, breaklines]{c}
int kill_process(char *process_name_or_id)
{
    if(string_to_int(process_name_or_id) == -1) {
        struct dirent *dirent;
        pid_t pid = 0;
        char *proc_dir_name = "/proc/";
        DIR *dir = opendir(proc_dir_name);
        if(dir == NULL) {
            exit(EXIT_FAILURE);
        }
        char    *buffer = NULL;
        char    *comm_file_path = NULL;
        char    *full_proc_path = NULL;

        FILE    *fp = NULL;

        bool    process_found = false;

        int count = 0;
        while((dirent = readdir(dir))!= NULL) {
            if(dirent->d_type == DT_DIR) {
                count++;
                if(strcmp(dirent->d_name, ".") == 0 || strcmp(dirent->d_name, "..") == 0) {
                    continue;
                }
                if(!is_numeric(dirent->d_name)) continue;

                size_t full_proc_pathLength = strlen(proc_dir_name) + strlen(dirent->d_name) + 1;
                full_proc_path = malloc(full_proc_pathLength);
                if(full_proc_path == NULL) {
                    goto cleanup;
                }
                snprintf(full_proc_path, full_proc_pathLength ,"%s%s", proc_dir_name, dirent->d_name);
                if((opendir(full_proc_path)) != NULL ) {

                    pid = (pid_t)atoi(dirent->d_name);

                    size_t comm_file_pathLength = strlen(proc_dir_name) + strlen(dirent->d_name) + strlen("/comm") + 1;
                    comm_file_path = malloc(comm_file_pathLength);
                    if(comm_file_path == NULL) {
                        goto cleanup;
                    }
                    snprintf(comm_file_path, comm_file_pathLength,"%s%s/comm", proc_dir_name, dirent->d_name);

                    fp = fopen(comm_file_path, "r");
                    if(fp == NULL) {
                        goto cleanup;
                    }

                    long filesize = 0;
                    while (getc(fp) != EOF)
                        filesize++;
                    fseek(fp, 0, SEEK_SET);

                    const size_t          BUFFER_SIZE = ((sizeof(char) * filesize) +1);

                    buffer = malloc(BUFFER_SIZE);
                    if(buffer == NULL) {
                        goto cleanup;
                    }

                    fgets(buffer, BUFFER_SIZE, fp);

                    if(ferror(fp)) {
                        goto cleanup;
                        return -2;
                    }

                    if(strcmp(strip_non_alpha(buffer), strip_non_alpha(process_name_or_id)) == 0) {
                        if(kill(pid, 9) == 0) {
                            fprintf(stderr, "killed %s with PID: %i", process_name_or_id, pid);
                            fflush(stderr);
                            process_found = true;
                        }
                        goto cleanup;
                    }
                }
            }
        }
        if(!process_found) {
            fprintf(stderr, "Could not find process '%s'\n", process_name_or_id);
        } 
    cleanup: 
        if(buffer) free(buffer);
        if(comm_file_path) free(comm_file_path);
        if(full_proc_path) free(full_proc_path);
        if(fp) fclose(fp);
        // the directory was not closed in every secenario
        if(dir) closedir(dir);
        return 0;
    }
    const pid_t pid = string_to_int(process_name_or_id);
    kill(pid, 9);
    fprintf(stderr,"killed %i\n", pid);
    fflush(stderr);
    return 0;
}
\end{minted}
\captionof{listing}{sourcecode implementation of the \funcname{kill\_process} function. \newline
  \textit{(some comments have been removed for clarity, can be found in the repository)}}
\label{mnt:kill}


\newpage
\usubsection{Execute Command - Input Redirection}
\begin{listing}[!htb]
\begin{minted}[linenos]{c}
int saved_stdin = dup(STDIN_FILENO);
int do_input_redirection = 0;
int fd_in = 0;

int input_red_pos = find_shell_operator("<", args);

if (input_red_pos > 0 && args[input_red_pos + 1] != NULL) {
    // < does not check if the file in the argument is a valid file,
    // it will redirect anything
    char        *filename = args[2];
    FILE        *file = fopen(filename, "r");

    if(file == NULL) {
        fprintf(stderr, "Error: %s is not a valid file.", filename);
    }

    if (feof(file) || ferror(file)) {
        perror("oshell: fread() failed");
        exit(EXIT_FAILURE);
    }

    // open filebased pipeline channel for file 'filename' in read only
    // 0 = stdin | 1 = stdout | 2 = stderr

    fd_in = open(filename, O_RDONLY);

    close(STDIN_FILENO); // we close stdin
    // we duplicate fd, into stdin
    if (dup2(fd_in, STDIN_FILENO) == -1) {
        perror("oshell: dup2 error");
        close(fd_in);
        return -1;
    }
    args[input_red_pos] = NULL; // remove '<' from the command

    do_input_redirection = 1;
}
\end{minted}
\caption{sourcecode showing the implementation of
input redirection}
\label{mnt:inputRedirection}
\end{listing}

\newpage
\usubsection{Execute Command -  Output Redirection}
\begin{minted}[linenos, breaklines]{c}
int saved_stdout = dup(STDOUT_FILENO);
int do_output_redirection = 0;
int fd = 0;

int truncate_pos = find_shell_operator(">", args);
int append_pos = find_shell_operator(">>", args);
// redirect stdout to the file
if ((append_pos > 0 && args[append_pos + 1] != NULL) || (truncate_pos > 0 && args[truncate_pos + 1] != NULL)) {
    // > truncate (overwrite) ; >> append

    if (args[truncate_pos + 1] == NULL || args[append_pos + 1] == NULL)  {
        fprintf(stderr, "Error: missing filename after > / >> \n");
        return -1;
    }

    if (args[truncate_pos + 2] != NULL || args[append_pos + 2] != NULL) {
        fprintf(stderr, "Error: too many arguments after > / >> \n");
        return -1;
    }

    char *filename = NULL;
    // if (append_pos != -1) { append_pos } else truncate_pos
    int pos = (append_pos != -1) ? append_pos : truncate_pos;
    if (pos != -1) {
        filename = args[pos + 1];
    }

    int flags;
    if (strcmp(">>", (args[append_pos])) == 0) {
        // inplace bitwise OR (x |= y ; x = x | y)
        // add flag O_APPEND to flags
        // &= ~xyz (remove xyz)
        flags = O_APPEND | O_WRONLY | O_CREAT; // create if needed, append if exists
    } else {
        flags = O_TRUNC | O_WRONLY | O_CREAT; // create if needed, truncate if exists
    }

    fd = open(filename, flags, 0644);
    if(fd == -1) perror("oshell: open() failed");

    if (dup2(fd, STDOUT_FILENO) == -1) {
        perror("oshell: dup2 error");
        close(fd);
        return 1;
    }

    // this is to remove the operator and filename from the command
    args[pos] = NULL;
    args[pos + 1] = NULL;

    do_output_redirection = 1;
}
\end{minted}
\captionof{listing}{sourcecode showing the implementation of
output redirection}
\label{mnt:outputRedirection}


\newpage
\usubsection{Utils - Find a shell operator}
\begin{minted}[linenos,breaklines]{c}
int find_shell_operator(char* operator, char **args) {
    for (int i = 0; args[i] != NULL; i++) {
        if (strcmp(args[i], operator) == 0) {
            return i;
        }
        if (args[i][0] == operator[0]) {
            return i;
        }
    }
    return -1;
}
\end{minted}
\captionof{listing}{sourcecode showing the implementation of
the \funcname{find\_shell\_operator} function}
\label{mnt:findOperator}

\newpage
\usubsection{Parsing Utils - Build a new string}
\begin{minted}[linenos]{c}
char *build_quote_string(char **arg, int start_arg, int start_pos, int end_arg, int end_pos) 
{
    int total_length = 0;

    for(int i = start_arg; i < end_arg; i++) {

 
        // this is needed otherwise, the code will try to execute the command provided e.g.
        // echo "foo bar baz" | wc -w just like that. 
        // this would output foo bar baz | wc -w /*
        /* We only want to skip the first character in the first word, 
         * since this is the outer quote that we want to remove
         * */
        int start_j;
        if(i == start_arg) start_j = start_pos + 1;
        else start_j = 0;

        int end_j;
        if (i == end_arg) end_j = end_pos;
        else end_j = strlen(arg[i]);

        for(int j = start_j; j < end_j; j++) total_length++; // amount of characters 
        if(i < end_arg) total_length++; // add N (amount of args) - 1 spaces for between each word
    }

    char *result = malloc((sizeof(char*) * total_length)  + 1);
    if(result == NULL) perror("oshell: memeory allocation for string result failed");
    int index = 0;

    /*
     * Building the string
     * This is basically the same as the loop before we make start_j and end_j 
     * and then loop from start to end, adding the characters one by one to the 
     * result string and null terminating it.
     * */

    for(int i = start_arg; i <= end_arg; i++) {
        int start_j;
        if(i == start_arg) start_j = start_pos + 1;
        else start_j = 0;

        int end_j;
        if (i == end_arg) end_j = end_pos;
        else end_j = strlen(arg[i]);

        for(int j = start_j; j < end_j; j++) result[index++] = arg[i][j];

        // add a space after each word
        if(i < end_arg) {
            result[index++] = ' ';
        }
    }
    return result;
}
\end{minted}
\captionof{listing}{sourcecode showing the implementation of the
  \funcname{build\_quote\_string} function}
\label{mnt:bqString}

\newpage
\usubsection{Piping}
\begin{minted}[linenos]{c}
int pipe_redirection(char **args) 
{
/*
     * To support more then one, pipe for "advanced" commands, there needs to be a way to track the amount of pipes we have
     * for later redirection, tracking of the multiple pipe positions, and the spliting of the commands
     * */
    int pipes_amount = 0;

    for (int i = 0; args[i] != NULL; i++) {
        if(strcmp("|", args[i]) == 0) pipes_amount++;
    }


    if(pipes_amount == 0) return -1; // no pipes found, however

    
    int pipe_pos[pipes_amount];
    int pipe_index = 0;
    for (int i = 0; args[i] != NULL; i++) {
        if (strcmp(args[i], "|") == 0) {
            pipe_pos[pipe_index++] = i;
        }
    }

    char ***commands = malloc((pipes_amount + 1) * sizeof(char**));
    char **paths = malloc((pipes_amount + 1) * sizeof(char*));

    int MAX_ARGS = 0;
    for(int i = 0; args[i] != NULL; i++) {
        MAX_ARGS++;
    }

    int start = 0;
    int end = 0;
    int cmd_id = 0;
    for (cmd_id = 0; cmd_id <= pipes_amount; cmd_id++) {
        if(cmd_id < pipes_amount) {

            commands[cmd_id] = malloc(MAX_ARGS * sizeof(char*));
            if (!commands[cmd_id]) {
                perror("oshell: memory allocation for command[cmd_id] failed");
            }
            memset(commands[cmd_id], 0, MAX_ARGS * sizeof(char*));

            end = pipe_pos[cmd_id];
            // create the commands from [cmd_id][i] = args[i]
            for(int i = 0; i < end; i++) {
                /*
                 * Since commands[cmd_id][i] is not initialized and only
                 * command[cmd_id] we need to allocate for each string in the array
                 * */
                if(args[i] != NULL) {
                    commands[cmd_id][i] = malloc(strlen(args[i]) + 1);
                    strcpy(commands[cmd_id][i], args[i]);
                }
                                }
            commands[cmd_id][end] = NULL;
        } else {
            commands[cmd_id] = malloc(MAX_ARGS * sizeof(char*));
            if (!commands[cmd_id]) {
                perror("oshell: memory allocation for command[cmd_id] failed");
            }
            memset(commands[cmd_id], 0, MAX_ARGS * sizeof(char*));

            int j = 0;
            do {
                j++;
            }while (args[j] != NULL);
            start = end + 1; // Skip the pipe
            end = j;
            for(int i = start; i < end; i++) {
                // using strdup we dont need to manually allocate memory for 
                // commands[cmd_id][xyz] since strdup handles that
                if(args[i] != NULL) {
                    // shit fuck why is i - start needed ??? DOES NOT WORK WITHOUT
                    commands[cmd_id][(i - start)] = my_strdup(args[i]);
                }
            }
            commands[cmd_id][end - start] = NULL;
        }
        size_t path_length = strlen("/usr/bin/") + strlen(commands[cmd_id][0]) + 1;
        paths[cmd_id] = malloc(path_length);
        if (!paths[cmd_id]) { perror("oshell: piping()"); exit(-1);}
        snprintf(paths[cmd_id], path_length, "/usr/bin/%s", commands[cmd_id][0]);        

    }
 
    /*
     * pipe() returns two file descriptors, fd[0] is open for reading, fd[1] is open for writing
     * ouput of fd[1] is input for fd[0].
     *
     * Since pipes are a point-to-point connection, one for every command, so we need n-1 pipes if n is the amount of commands we have
     * */

    int fd[pipes_amount][2];

    for(int i = 0; i < pipes_amount; i++) {
        if (pipe(fd[i]) == -1) {
            perror("oshell: fd ");
        }
    }

    /*
     * We need to track, what gets redirected where
     * "This is the first command, only redirect STDOUT"
     * "This is a middle command, redirect both"
     * "This is the last command, only redirect STDIN"
     * */

    // save the stdin/stdout, to later restore 
    int saved_stdout = dup(STDOUT_FILENO);
    int saved_stdin = dup(STDIN_FILENO);

    int f = 0;
    for(int i = 0; i < cmd_id; i++) {

        if ((f = fork()) == 0) {
            // child: setup the redirections
            if(i == 0) { // first command, only redirect stdout
                if(dup2(fd[i][1], STDOUT_FILENO) == -1) {
                    fprintf(stderr, "Child %d: dup2 failed\n", i);
                    _exit(EXIT_FAILURE);
                }
                for(int j = 0; j < pipes_amount; j++) {
                    close(fd[j][0]);
                    close(fd[j][1]);
                }
            } else if (i == cmd_id - 1) { // last command only redirect stdin
                if(dup2(fd[i-1][0], STDIN_FILENO) == -1) {
                    close(fd[i-1][0]);
                    _exit(EXIT_FAILURE);
                }
                for(int j = 0; j < pipes_amount; j++) {
                    close(fd[j][0]);
                    close(fd[j][1]);
                }
            } else { // not first nor last, redirect stdin and stdout
                     // cmd0 ---> pipe[0] ---> cmd1 ---> pipe[1] ---> cmd2
                if(dup2(fd[i-1][0], STDIN_FILENO) == -1 || dup2(fd[i][1], STDOUT_FILENO) == -1) {
                    _exit(EXIT_FAILURE);
                }
                for(int j = 0; j < pipes_amount; j++) {
                    close(fd[j][0]);
                    close(fd[j][1]);
                }
            }
            if(commands[i] == NULL) {
                fprintf(stderr, "ERROR: commands[%d] is NULL\n", i);
                _exit(EXIT_FAILURE);
            }

            if (execv(paths[i], commands[i]) == -1) {
                perror("execv() failed");
                free(paths);
                free(commands);
                for(int i = 0; i < pipes_amount; i++) {
                    close(fd[i][0]);
                    close(fd[i][1]);
                }
                _exit(EXIT_FAILURE);
            }
            fprintf(stderr, "Child %d: ERROR - this should never print!\n", i);
            } else if (f == -1) {
                perror("oshell: fork() failed"); 
                return -1; 
            }
            // parent continues to next iteration
        }
    for(int j = 0; j < pipes_amount; j++) {
        close(fd[j][0]);
        close(fd[j][1]);
    }

    for(int i = 0; i < cmd_id; i++) {
        int status;
        pid_t result = waitpid(-1, &status, WNOHANG);  // Non-blocking wait
        if (result == 0) {
            continue;
        } else if (result > 0) {
        }
    }

    // restore stdin/stdout
    dup2(saved_stdin, STDIN_FILENO);
    dup2(saved_stdout, STDOUT_FILENO);

    if(close(saved_stdin) != 0) perror("oshell: piping()");
    if(close(saved_stdout) != 0) perror("oshell: piping()");

    free(paths);
    free(commands);

    return 0;
}
\end{minted}
\captionof{listing}{sourcecode showing the implementation of the
\funcname{pipe\_redirection} function}~\label{mnt:pipeRedirection}

\end{document}
